\begin{abstract}
  ARINC 429 is the most common data bus in use today in civil avionics. Despite its widespread use, it is not up to par with modern security requirements. 
  %as it did not undergo any major revisions since the 1980s. 
  Specifically, the protocol lacks any form of source authentication. 
  A technician with physical access to the bus is able to replace a transmitter by a rogue device, and the receivers will accept its malicious data as they have no method of verifying the authenticity of messages.
  
  Updating the protocol would close off security loopholes in new aircrafts but would require thousands of airplanes to be modified. For the interim, until the protocol is replaced, we propose the first intrusion detection system that utilizes a hardware fingerprinting approach for sender identification for the ARINC 429 data bus. Our approach relies on the observation that changes in hardware, such as replacing a transmitter or a receiver with a rogue one, modify the electric signal of the transmission. By exploiting this observation, we are able to detect technician attacks. Our proposed method only requires the attachment of a standard-compliant monitoring unit to the bus. 
  
  Because we rely on the analog properties, and not on the digital content of the transmissions, we are able to detect a hardware switch as soon as it occurs, even if the data that is being transmitted is completely normal. Thus, we are able to preempt the attack before any damage is caused.
  
  In this \iftoggle{paper} {paper} {work} we describe the design of our intrusion detection system, and evaluate its performance against different adversary models. Our analysis includes both a theoretical Markov-chain model, and an extensive empirical evaluation. For this purpose, we collected a data corpus of ARINC 429 data traces, which may be of independent interest since, to the best of our knowledge, no public corpus is available.
  We find that our intrusion detection system is quite realistic: e.g., it achieves near-zero false alarms per second, while detecting a rogue transmitter in under 200ms, and detecting a rogue receiver in under 3 seconds. In other words, technician attacks can be reliably detected during the pre-flight checks, well before the aircraft takes off.
\end{abstract}