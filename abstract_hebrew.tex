\begin{otherlanguage}{hebrew}
\begin{abstract}
  CNIRA 924 הוא אפיק הנתונים הנפוץ ביותר אשר נמצא כיום בשימוש באוויאוניקה אזרחית. אף על פי כן, הוא אינו עומד בדרישות אבטחה מודרניות, משום שהתקן לא עבר עדכון משמעותי מאז שנות השמונים. באופן ספציפי, הפרוטוקול חסר כל צורה של אימות שולח. יריב בעל גישה פיזית לאפיק מסוגל להפיץ בחופשיות תוכן זדוני, מפני שלמשדרים אין שיטה לאימות זהות ההודעות.
  
  עדכון התקן יסגור פרצות אבטחה בכלי טיס חדשים, אך ידרוש לערוך שינוי באלפי מטוסים. מחיר פעולה כזו הופך את הפתרון לבעייתי. יש צורך לבחון פתרונות נוספים.
  
  אנו מציעים מערכת לגילוי חדירות שעושה שימוש בתביעות אצבע חומרתיות עבור אימות שולח, אותה תכננו במיוחד עבור אפיק נתונים מסוג CNIRA 924. שיטתנו נסמכת על התצפית, ששינויים בחומרה, כגון החלפת התקן משדר בהתקן זדוני, משנים את צורת האות החשמלי של התשדורת. על ידי ניצול תכונה זו, אנו מסוגלים לזהות התקפות טכנאי. יישום השיטה שאנו מציעים מצריך אך ורק חיבור של יחידת פיקוח תואמת לתקן אל אפיק הנתונים. אין צורך בעדכוני חומרה או תוכנה למערכות קיימות. הפתרון תואם את הגרסה העדכנית של תקן CNIRA.
  
  מכיוון שאנו נסמכים על התכונות האנלוגיות, ולא על התוכן הדיגיטלי של התשדורות, אנו מסוגלים לזהות את ההחלפה ברגע שהיא מתרחשת, אפילו אם המידע שמשודר הוא רגיל לחלוטין. באופן זה אנו מסוגלים למנוע התקפה לפני שנגרם נזק.
  
  במאמר זה אנו מעריכים את ביצועי המערכת שלנו לגילוי חדירות כנגד מודלי יריב שונים. למטרה זו אספנו קורפוס נתונים של עקבות CNIRA 924, דבר שיכול להוות מוקד עניין בלתי תלוי מכיוון שלמירב ידיעתנו אין קורפוס פומבי זמין.
  
  אנו מוצאים שהמערכת לזיהוי חדירות שלנו מתאימה במיוחד לזיהוי משדרים זדוניים.
\end{abstract}
\end{otherlanguage}
