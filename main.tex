\documentclass[conference]{IEEEtran}
% Add the compsoc option for Computer Society conferences.
%
% If IEEEtran.cls has not been installed into the LaTeX system files,
% manually specify the path to it like:
% \documentclass[conference]{../sty/IEEEtran}

\pagestyle{plain}

\usepackage{cite}
\usepackage{amsmath,amssymb,amsfonts}
\usepackage{algpseudocode} % changed from &&&& 
\usepackage{algorithm} % changed from &&&& 
\usepackage{graphicx}
\usepackage{textcomp}
\usepackage{xcolor}
\usepackage{url} % to enable usage of long URLs ion the bib (otherwise they escape the line)
\usepackage{csvsimple}
\usepackage{tabularx}
\usepackage{xfrac}

\usepackage{hyperref}
\hypersetup{
    colorlinks=true,
    allcolors=.,
    urlcolor=blue,
}

\title{Hardware Transmitter Fingerprinting in the ARINC 429 Avionic Bus}
%\author{Avishai Wool, Nimrod Gilboa Markevich}
\author{Anonymous}

\begin{document}

\maketitle

\begin{abstract}
  preemptive anomaly detection
    bla
    
    e.g., 
    
    i.e.,
    
    etc.,
    
    et al.\ said
    
\end{abstract}


\section{Introduction}
  % (I wanted to start with a bang, but couldn't build up the argument. Maybe in the abstract)
  % In recent years we have experienced increasing awareness of the threats of cyber-physical attacks and of the importance of proper cybersecurity. While in the past, it was acceptable to rely on limiting physical access as a security countermeasure, this is no longer considered true. Security experts advise us that everything can be hacked. Conventional wisdom nowadays is to prepare for the worst.
  
  ARINC 429 is a prominent standard for wired intra-vehicle communication in commercial aircrafts \textcolor{gray}{[cite spec]}. Most active and retired airplanes contain ARINC buses \cite{fuchs2012evolution}, interconnecting the many digital systems that are necessary for the operation of an aircraft: sensors, radars, engines, cockpit controls and more.
  
  Safety and reliability are key objectives in avionics \cite{fuchs2012evolution}. Therefore, the main requirements of airborne subsystems are high determinism and low response times \cite{thanthry2005aviation}. ARINC 429 was designed accordingly. Security on the other hand, as we understand it today, was not primary concern. At the time of the protocol's release in 1977, awareness of the threat of cyber-physical attacks was not as wide-spread as it is today. ARINC 429 was designed without any security features, such as encryption or source authentication, that are now considered basic. In the years that passed the importance of proper cybersecurity was demonstrated in numerous fields, from industrial networks \textcolor{gray}{[cite favorite Stuxnet paper]} to cars \cite{miller2015remote} \textcolor{gray}{more? smart homes?}. However, since 1980 there were no major revisions of the ARINC 429 standard \cite{18937420070101}. Currently, as ARINC 429 has no mechanism for source authentication, once an attacker has gained physical access to bus, any data they transmit will be accepted.
  
  \textcolor{gray}{AFDX is the successor to ARINC 429. It is ethernet based. So IPSec or other modern solutions can be applied. Still, to quote from \cite{fuchs2012evolution}: ``However, 429 will most likely not simply vanish; it will still be used in scenarios where simple signaling is sufficient, and in latency critical scenarios. It is a proven and extremely reliable technology and thus is also used as fall-back network for the AFDX network, e.g. in the Airbus A380.''}
  
  One way to add authentication without an industry-wide update of the protocol is to implement it on a higher layer \textcolor{gray}{(clear?)}. Unfortunately, in ARINC 429 there are only 19 data bits in an message. This is typically insufficient for a secure implementation of message code authentication (MAC).
  
  Intrusion detection systems (IDS) are often employed to retrofit security into similar systems that were designed without security in mind, such as CAN bus in automobile systems \cite{}. An IDS is a software or device that continuously compares the observed behavior of the system to the expected behavior and raises an alarm if anomalies are detected. However, ARINC 429 messages do not include a source identifier (authenticated or otherwise), so an IDS than relies solely on analyzing message content is incapable of identifying the sender of that message. In order to implement source authentication, the electrical signal has to be examined. The guiding principle is that every transmitter is unique, even those of the same model and maker, due to minor defects in production, component tolerances etc., which manifest in the electrical signal. Therefore, every signal has inimitable characteristics that can be used to identify the sender. The process of learning to associate a signal to a transmitter is called hardware fingerprinting.
  
  We propose the use of hardware fingerprinting in order to imbue ARINC 429 buses with source authentication capabilities. Applying the method only requires the attachment of a standard-compliant monitoring unit to the bus. This method does not require hardware or software updates to existing systems and is compliant with the current version of the ARINC standard.
  
  In this paper we design, and evaluate the performance of ARINC 429 HW fingerprinting. We compare different feature sets, explore the effects of message content on performance, explore the ability to distinguish between devices from different vendors and between devices of the same model. We explore the effect of the receiver and transmission line on performance. We explore the effect of additional listeners on performance.
  
  \textcolor{gray}{Paper structure.}

% \subsection{Motivation}
%   \color{gray}
%   Advantage 1: ...Of course, the attacker is aware of the possibility of a cyber defence systems monitoring the communication on the bus. As a countermeasure, they will attempt to mask the offensive data as data naturally occurring in the system. This type of operation is meant to make it hard for defence systems to detect intruders solely based on the digital data. Advantage 2: (Is it true to say, that other algorithms require a sequence of words, while ours can make a decision based on a single word? Maybe other systems can do that too, or maybe our system should use more than one word for robustness?)
%   No need to change the protocol or the systems. Does not increase computational demands or communication overhead.
%   Backward compatible.
%   \color{black}
%   \textcolor{gray}{(Should we address lack of authentication here (in addition to mentioning it in the motivation section)?)}

\subsection{Related Work}
  To the best of our knowledge, this is the first academic research to suggest hardware fingerprinting in ARINC 429. However, hardware fingerprinting was explored previously in a number of different domains: Ethernet \cite{kohno2005remote, uluagac2013passive, gerdes2012physical}; wireless radio \cite{ellis2001characteristics, hall2003detection, hall2005radio, ureten2007wireless, brik2008wireless, xu2015device}; smartphone accelerometers, gyroscopes, microphones and cameras \cite{lukavs2006digital, li2010source, dey2014accelprint, bojinov2014mobile, das2014you, das2016tracking}.
  
  One domain in particular is of special interest to us: controller area network (CAN bus) \cite{}, the most commonly used standard for in-vehicle communication in the automotive industry. ARINC 429 and CAN bus have a lot in common: Both are protocols for wired local area networks. Both are meant to be used in a static topology. Both share similar bit rates (up to 100 Kbits/sec in ARINC, up to 1 Mbits/sec in CAN) and similar word lengths (32 bits in ARINC, up to 128 bits in CAN \textcolor{gray}{without bit stuffing, 140 with}). Both were formulated more than 30 years ago, and both were not designed for security but rather for safety, and lack source authentication.
  In recent years a number of successful cyber attacks were demonstrated on cars \cite{}, motivating researchers to search for new ways to hinder attacks.
  
  % like ARINC 429, straightforward authentication tools like HMAC are an unlikely solution \textcolor{gray}{(elaborate?)}. 
  
  In \cite{cho2016fingerprinting} the authors propose using the measured arrival time of periodic messages to estimate characteristics of a transmitter's internal clock, which are then used as fingerprints. It has been demonstrated in \cite{sagong2018cloaking} that timing features can be emulated by an attacker, and therefore might not be a reliable means for identification.
  In \cite{murvay2014source} the authors have demonstrated that transmitters of CAN messages can be identified by the signal's electrical characteristics. They propose using the CAN message ID field of the electric signal as a fingerprint. They also observed that the electrical characteristics remain stable for a period of months in a lab setup.
  
  In \cite{choi2018identifying} the authors suggest using time domain and frequency domain features extracted from the CAN ID field as fingerprints. They perform feature selection using the mutual information criterion in order to reduce the number of used features. ML techniques are utilized for classification (SVM, NN and BDT).
  
  In \cite{cho2017viden} the authors devise a method for generating fingerprints based on the order statistics of voltage levels. The algorithm adapts to power supply fluctuations and temperature changes by constantly updating the fingerprints based on new samples.
  
  In \cite{choi2018voltageids} the authors construct the fingerprints by extracting time domain and frequency domain features from selected parts of a message. They perform feature selection with sequential forward selection, and SVM and BDT for classification. Incremental learning techniques \cite{diehl2003svm} are employed in order to compensate for temporal changes of the characteristics.
  
  In \cite{kneib2018scission}, when a frame arrives, the authors first build a number of artificial signals from that frame. The artificial signals are made by cutting the signal and concatenating parts that share a similar behavior: positive slope transient, negative slope transient and stable positive voltage. For each of these three artificial signals, a set of time domain and frequency domain features are extracted.
  
  CAN bus and ARINC 429 use different line protocols, therefore methods presented in the above papers cannot be directly applied to our problem without change.
  
  A key difference between the two protocols is the bus topology. In CAN bus dozens of transceivers may share one bus. The main threat CAN papers are dealing with is device hijacking, where one ECU is remotely hacked, and starts to message the messages of another ECU on the same bus. Since during normal operation of a car, devices do not spontaneously join or leave the network, all the devices are known to the defender in advance. This scenario naturally fits into a multiclass classification setting, where a message needs to be identified as belonging to one of many known classes. In ARINC 429 on the other hand, only one transmitter is allowed on a line. Only the guarded transmitter is known to the defender beforehand. The task is to categorize a message as either legitimate or as an anomaly (attack). Multiclass classification algorithms are not suitable for this task, because only samples from the legitimate class can be obtained for training. Instead we need to use novelty detection algorithms \cite{pimentel2014review}. It is worth noting that some papers (\cite{choi2018identifying}, \cite{choi2018voltageids}) do extend their algorithms to handle detection of unknown transmitters. However, this is still not their primary objective.
  
\subsection{Contributions}
  It is important to note that HW fingerprinting cannot replace the use of data-driven \textcolor{gray}{(correct usage?)} IDSs, as HW fingerprinting only detects attacks, where one transmitter sends messages, that are normally sent by another transmitter.
  
  \textcolor{gray}{lack of recordings of real attack proposes a challenge. we propose a method. demonstrate the importance of maintaining equal conditions (topology, receiver) when evaluating.}

\section{Preliminaries}
% stuff we didn't invent or discover but the reader needs to know
\subsection{The ARINC 429 Standard}
  ARINC Specification 429 \cite{}, also named ``Mark 33 Digital Transfer System (DITS)'', is a standard of the avionics industry. It defines a protocol for the communication between avionics system elements over a local area network. First published in 1977 by Aeronautical Radio, Inc., it has since become one of the most widely used data bus protocols in commercial aircrafts \cite{}. The protocol encompasses different layers: from the physical requirements, through the electronic characteristics of the signal, data format and ending with a file transfer technique.

  We continue with a short description of those parts of the specifications, which are relevant to this paper. Data is transmitted over a single twisted and shielded pair of wires. The cable shield is be grounded on both ends. The lines are named Line A and Line B. Differential signaling is used, meaning that the signal is the voltage difference from Line A to Line B, rather than the difference from one wire to ground. Bipolar return-to-zero (BRTZ) modulation is used as a line protocol. BRTZ is a tri-level state modulation: we refer to the three voltage levels as ``HI'', ``LO'' and ``NULL''. A binary 1 is encoded as a positive voltage pulse ``HI'', and a binary 0 is encoded as a negative voltage pulse ``LO''. In between transmissions, the voltage drops to 0V, ``NULL''. Every ``HI'' and every ``LO'' are preceded and are followed by a ``NULL'', even if repeating bit values are transmitted consecutively. The differential output voltage from line A to line B is $10V \pm 1$ in ``HI'' mode, $0 \pm 0.5$ in ``NULL'' mode and $-10V \pm 1$ in ``LO'' mode.  Figure \ref{fig:word_example} shows a recording of a transmission on an ARINC 429 data bus.
  
  Data is transmitted in words that are 32-bit long. The bits are transmitted in the following order, from first to last: 8, 7, ..., 2, 1, 9, 10, ..., 32. This order is a legacy from older systems. In this paper, words are interpreted as though an MSB-first transmission order is in place.
  
  Data is transmitted unidirectionally from a single transmitter to up to 20 receivers. Only one transmitter is allowed on the bus - a separate bus is required for each transmitter. Since there is only one transmitter on each bus, there is no sender ID field in ARINC messages.
   
  The protocol allows a choice of one of two bit rates: Slow, at 12.0 to 14.5 Kbits/sec, and fast, at 100 Kbits/sec. The bit rate on a bus is fixed and maintained within \%1. The signal is self-clocking.
  
  The specification does not address the issue of authentication. \textcolor{gray}{(expand)}
   
  MIL-STD-1553 \cite{} is the military bus standard alternative of ARINC 429.
  
  \begin{figure}[t]
    \centering
    \includegraphics[width=1.0\linewidth, angle=0]{Traces/word.png}
    \caption{ARINC 429 bus showing the voltage difference between twisted pair for a 32 bit word}
    \label{fig:word_example}
  \end{figure}
  
\subsection{The Adversary Model}
  Out method is designed to guard against ``technician attacks''. This type of attack involves an adversary that has brief physical access to the system. Such an adversary is able to replace a legitimate transmitter with a malicious one. During an initial dormant phase of the attack, the new device imitates the behavior of the legitimate transmitter, transmitting data exactly as requested, in order to remain hidden. Only at a later time, the attack moves on to its active phase. During this phase the malicious transmitter sends out messages which disrupt the work of the system, and in extreme cases causes irreversible damage to the electronic or physical components.
  
  % DELETE THIS PARAGRAPH?
  % An attacker may or may not posses knowledge of the bus topology - i.e., number of receivers, distances, and models of connected devices. The attacker will try their best to mimic the characteristics of the original transmitter. In the most case for the defenders, the attacker will use a transmitter which is identical (same manufacturer and model) to the compromised transmitter. (True statement? It is theoretically possible to use a massive signal generator, or maybe a dedicated device that can adjust itself to the real transmitter).

  The attacker may have prior knowledge of the hardware and topology of the attacked system. The reverse is not true: As defenders, we have no prior knowledge of what the attacker's hardware might be. However, we do assume that the adversary will use a commercial off-the-shelf transmitter. Therefore, we used commercial transmitters in our tests.

  The monitoring system we propose has to be attached to the bus it is guarding. During a training period it samples the bus and learns the transmitter's characteristics. We assume that during this time only legitimate devices are present on the bus. We further assume that access to the monitoring system is restricted, so that only authorized personnel are able to trigger the training mechanism. This restriction is in place in order to prevent an attacker from retraining the monitoring system after switching the legitimate transmitter by the malicious one. 
  
  As our method operates on a word by word basis, it is unable to raise the alarm until after the first malicious word is already received. We find this is a reasonable limitation, since the amount damage an attacker can cause with a single word is limited. \textcolor{gray}{(unnecessary?)}

\subsection{Adding another transmitter - fails}
  It is important to remember that the ARINC 429 bus is designed to allow exactly one transmitter. Connecting two transmitters to the same bus violates the protocol, therefore an adversary cannot simply add a transmitter to the bus. Furthermore, it is not possible to turn a receiver into a transmitter by software only, since its wiring does not permit it. Thus, the adversary must use physical access to the equipment in order to mount the attack. The adversary needs to make sure that the legitimate transmitter is disconnected, before connecting the rogue transmitter. Otherwise, there is a risk that data will fail to be delivered. In fact, when we naively connected two transmitters to the same bus, the peak to peak voltage dropped by half, and the legitimate communication on the bus failed. While we don't assert that this will always be the case, it serves as a cautionary anecdote for adversaries. Further, an adversary may possibly construct special hardware that would allow the bus to function with two or more transmitters, for example by disconnecting the legitimate transmitting during transmissions of the rogue transmitter, but the fact remains that standard commercial components would not suffice.

\section{Signal Segmentation}
  Our method aims to rely solely on the physical characteristics of the hardware, and aims to be completely agnostic to the transmitted data. In order to achieve this goal, we divide each word into sub-bit non-overlapping segments.
  
  In a BRTZ line protocol, each bit comprises of 4 distinct segments. For example, a 1 bit starts with a transition up from NULL to HI, then a plateau on HI, then a transition down from HI back to NULL, and finally a NULL plateau. Furthermore, we observed 4 different variants of NULL, depending on the current and the next bit. All in all we identified 10 different segment types, as listed in Table \ref{tab:SegmentationLevels}.
  
  Thus, we split every 32-bit word into 127 segments. Note that there are only 127 segments, not 128, because the last bit is followed by a long NULL that lasts until the next word. We ignore this last segment, because it is unique.
  
  The segmentation is performed in the following manner. A segment starts where the voltage level of the signal rises above / falls below a certain threshold, and ends where it falls below / rises above another threshold. Four different thresholds are employed in order to produce a stabling hysteresis effect. We denote them as follows, and use them and their negative to define segment boundaries:
  
  \begin{align*}
    V_{l_1} = 2.0V \\
    V_{l_2} = 2.8V \\
    V_{h_1} = 8.0V \\
    V_{h_2} = 7.2V 
  \end{align*}
  
  Table \ref{tab:SegmentationLevels} shows the voltage levels used for each segment type. Figure \ref{fig:SegmentationTrace} shows an example of word segmentation. \textcolor{gray}{(I'll change the trace. The current figure is confusing).}
  
  \begin{table}
    \caption{Voltage Thresholds per Segment Type}
    \label{tab:SegmentationLevels}
    \centering
    \begin{tabular}{|c c c|} 
      \hline
      Segment & Starting Threshold & Ending Threshold \\ [0.5ex] 
      \hline\hline
      LO & falls below $-V_{h_1}$ & rises above $-V_{h_2}$ \\
      \hline
      HI & rises above $V_{h_1}$ & falls below $V_{h_2}$ \\
      \hline
      NULL, HI to HI & falls below $V_{l_1}$ & rises above $V_{l_2}$ \\
      \hline
      NULL, HI to LO & falls below $V_{l_1}$ & falls below $-V_{l_2}$ \\
      \hline
      NULL, LO to LO & rises above $-V_{l_1}$ & falls below $-V_{l_2}$ \\
      \hline
      NULL, LO to HI & rises above $-V_{l_1}$ & rises above $V_{l_2}$ \\
      \hline
      Up from LO & rises above $-V_{h_2}$ & rises above $-V_{l_1}$ \\
      \hline
      Up from NULL & rises above $V_{l_2}$ & rises above $V_{h_1}$ \\
      \hline
      Down from HI & falls below $V_{h_2}$ & falls below $V_{l_1}$ \\
      \hline
      Down from NULL & falls below $-V_{l_2}$ & falls below $-V_{h_1}$ \\
      \hline
    \end{tabular}
  \end{table}
  
  \begin{figure}[t]
    \centering
    \includegraphics[width=1.0\linewidth, angle=0]{Traces/segmentation.png}
    \caption{A segmentation example of the bits 001011}
    \label{fig:SegmentationTrace}
  \end{figure}
  
  \color{gray}
  why - data agnostic
  quarter-bits; how
  show graph of segmented signal (?)
  characteristics: ascent, over/under-shoot, return to zero and go down/up,
  etc...

  sampling rate + downsampling
  \color{black}
  
\section{The Hardware Fingerprinting Approach}
  An intrusion detection system which draws its information exclusively from the digital content of the transmitted messages will be unable to detect the rogue transmitter during the dormant phases. Only during an active phase of the attack is the transmitted data distinguishable from that of a legitimate transmission. The IDS we propose utilizes information from the analog domain. At the electronic level, the two transmitters differ even during dormant phases. The malicious hardware can be flushed out as soon as it begins signalling.
  Usually, before takeoff, the aircraft systems are checked for basic integrity. During this pre-flight operations the transmitter replacement will be detected.
  
\subsection{IDS Overview} \label{Overview}
  Our method operates on a word by word basis. When a new word is captured and tested for anomalies, we process it in several stages. This section provides an overview of these steps. In the subsequent sections each stage is explained in greater detail.
  
  \begin{enumerate}
    \item \textbf{[Acquisition]}
          We sample both lines of the bus at a sampling rate that is 50 times higher than the rate which is needed in order to extract the bit values. We used a sample rate of 5 MSa/s. The differential signal is received from subtracting the samples of line B from the samples of line A.
    \item \textbf{[Segmentation]}
          The word is split into 127 segments of 10 different types, based on voltage levels. The purpose of the segmentation is to eliminate the effect of the transmitted data, the content of the word, on the final decision of the anomaly detector.
    \item \textbf{[Feature Extraction]}
          We extract multiple features from each segment. 
    \item \textbf{[Anomaly Detection Per Segment]}
          The features from each segment are fed into a trained anomaly detector. Each \textit{segment} is marked as either ``normal'' or ``anomaly''.
    \item \textbf{[Voting]}
          A word is declared as an anomaly, if the number of ``anomaly'' segments exceeds a predetermined threshold.
  \end{enumerate}
  
\subsection{The Data Corpus}
  To the best of our knowledge, there is no publicly available data corpus that contains high rate samples of ARINC 429 protocol. We built our own data set, with the kind assistance of \textit{Astronautics C.A. LTD.} \cite{}.
  
  We sampled two types of transmitters: 1. A M4K429RTx test equipment from \textit{Excalibur Systems} \cite{}. The Excalibur equipment hosts two transmitters, on two boards which we label E1 and E2. 2. A HI-8597PSIF chip on HI-3220PQ Evaluation Board, manufactured by \textit{Holt Integrated Circuits INC.} \cite{}. We used one of two boards, each with 4 chips. Each chip has 1 transmitter and 2 receivers. We label the transmitters H{x}{y}, where x is the board number, 1 or 2, and y is the transmitter number from 0 to 3. The receivers are labeled in a similar fashion. H{x}{y}, where x is the board number, and y is the receiver number. We only used a subset of the available receivers. Either the Excalibur or one of the Holt boards communicated with a proprietary device manufactured by Astronautics \textcolor{gray}{(waiting for a part number from Astronautics)}. This device was not sampled as a transmitter. We label the receivers P1 and P2.
  
  For sampling we used a Keysight DSO9254A scope. All signals were sampled at 50Msa/s at a scope bandwidth of 25MHz. The probes are 500MHz, 10M\(\Omega\), 11pF. Each line was sampled individually. We further downsampled digitally by a factor of 10 to a rate of 5 MSa/s using a 30 point FIR filter with Hamming window.
  
  The transmitters and receivers were connected through a custom board that exposes the wires, which we fabricated for this purpose (see Figure \ref{fig:SetupImage}).
  
  \begin{figure}[t]
    \centering
    %Use 1 for the two columns
    \includegraphics[width=1.0\linewidth, angle=0]{Images/setup_3}
    \caption{The Holt evaluation board on the left, andthe fabricated connector board on the right}
    \label{fig:SetupImage}
  \end{figure}
  
  All devices transmitted the same data. 6 values of words were transmitted. Interpreting the words with MSB-first transmission order, the values are: \texttt{0x00000000}, \texttt{0xFFFFFFFF}, \texttt{0x55555555}, \texttt{0xAAAAAAAA}, \texttt{0x5A5A5A5A}, \texttt{0xA5A5A5A5}. Transmitting the same data on all devices eliminates the possibility that instead of learning the analog features our transmitter unintentionally will learn the encoded data.
  
  In addition to recording transmitter-receiver pairs, we recorded E1 and E2 transmitting to P1 and P2 respectfully, with different Holt devices attached as additional receivers. Table \ref{tab:RecordingsSummery} shows the different combinations of transmitter-receiver in our data set, and the number of words recorded for each combination.
  
  \begin{table}
    \caption{Distribution of Recorded Words in the Data Set}
    \label{tab:RecordingsSummery}
    \centering
    \begin{tabular}{|c c c|} 
      \hline
      Transmitter & Receiver & \#Words \\ [0.5ex] 
      \hline\hline
      E1 & P1 & 4920 \\ % There are actually 9840, but I use half to keep a balanced data set.
      \hline
      E1 & P1 \& H10 & 4920 \\
      \hline
      E1 & P1 \& H12 & 4920 \\
      \hline
      E1 & P1 \& H20 & 4920 \\
      \hline
      E1 & P1 \& H22 & 4920 \\
      \hline
      H10 & P1 & 4920 \\
      \hline
      H11 & P1 & 4920 \\
      \hline
      H12 & P1 & 4920 \\
      \hline
      H13 & P1 & 4920 \\
      \hline
      H20 & P1 & 4920 \\
      \hline
      H21 & P1 & 4920 \\
      \hline
      H22 & P1 & 4920 \\
      \hline
      H23 & P1 & 4920 \\
      \hline
      E2 & P2 & 4920 \\ % There are actually 9840, but I use half to keep a balanced data set.
      \hline
      E2 & P2 \& H10 & 4920 \\
      \hline
      E2 & P2 \& H12 & 4920 \\
      \hline
      E2 & P2 \& H20 & 4920 \\
      \hline
      E2 & P2 \& H22 & 4920 \\
      \hline
      H10 & P2 & 4920 \\
      \hline
      H11 & P2 & 4920 \\
      \hline
      H12 & P2 & 4920 \\
      \hline
      H13 & P2 & 4920 \\
      \hline
      H20 & P2 & 4920 \\
      \hline
      H21 & P2 & 4920 \\
      \hline
      H22 & P2 & 4920 \\
      \hline
      H23 & P2 & 4920 \\
      \hline
    \end{tabular}
  \end{table}
  
\subsection{Novelty Detection per Segment}
  As mentioned in the overview section [\ref{Overview}], after segmentation and feature  extraction, the next step in our algorithm is to perform per segment anomaly  detection. There are 10 types of segments, as detailed in Table \ref{tab:RecordingsSummery}. Because the segments are very different from each other \textcolor{gray}{(elaborate: in what respect?)}, we opted to train a different novelty detector for each type of segment.
  
  There are many outlier and novelty detection algorithms available in the literature such as K-Nearest Neighbors \cite{hautamaki2004outlier}, Mixture Models \cite{}, One-Class SVM \cite{}, Isolation Forest \cite{liu2008isolation}. An extensive review of various algorithms is presented in \cite{pimentel2014review}.
  
  For the novelty detection task, we chose to work with the Local Outlier Factor (LOF) by Breunig et al.\ \cite{breunig2000lof}. LOF was shown to work better then other algorithms for the task of network intrusion detection\cite{lazarevic2003comparative}. This fact, together with the available scikit-learn \cite{scikit-learn} python implementation, made it an appealing choice. Comparing different anomaly detection algorithms is beyond the scope of this paper.
  
  LOF is a density based outlier detection algorithm. According to the LOF algorithm, an outlier is defined as a data point (feature vector), whose local density is greater than the average of local densities of its neighbors by a large enough margin. A local density of a data point is the inverse of the average distance of the point from its neighbors.
  
  There are several hyper-parameters for the LOF algorithm. In all cases we used the default parameters provided by implementation. For the number of neighbors examined when calculating the LOF the default is 20. We used the Euclidean metric for the distance measure. The threshold on the local outlier factor that defines an anomaly is automatically set so 10\% of samples in the \textbf{training set} are outliers.
  
  We constructed a separate novelty detector for each type of segment. In this stage, each segment is fed individually into its appropriate LOF novelty detector. The LOF outputs its suggestion regarding the source of the segment, either `normal' or `anomaly'.
  
\subsection{Word-Based Anomaly Detection}
  We gather all the suggestions made by the different LOF detectors for all segments of the same word. The number of segments that have been identified as legitimate is subjected to a threshold. If it is greater than that threshold, the word is predicted to be `normal', otherwise, it is flagged as an `anomaly'.
  
%%%%inline formula $\sum_{i=0}^n x_i $

%%%displayed formula
%%%\[
%%%\sum_{i=0}^n x_i
%%%\]

%%%method 2: half-bits; how

%%%%%%%%%%%%%%%%%
\section{Feature Selection}

  \color{gray}
  advantages of feature selection (model size reduction, speed, accuracy?)
  \color{black}


\subsubsection{Scission Feature}

  \cite{kneib2018scission}

\subsubsection{domain-specific features}
  potential features
  \begin{itemize}
    \item ascent rate
    \item maximum and minimum in plateau
    \item ...
  \end{itemize}
 
 \subsection{The Benjamini Hochberg method}
 
 \cite{benjamini1995controlling}
 
  labeled data
 
  results: top X features  (bar chart)

%%%%%%%%%%%%%%%%%%
\section{Performance Evaluation}

\subsection{Methodology}
  In order to evaluate the performance of our algorithm, we performed a series of experiments. In each experiment we designate one of the our transmitters as the legitimate device to be guarded. In case the transmitter is an Excalibur (E1 or E2), all the other devices are in the role of attackers. In case the transmitter is a Holt (H10, ..., H13, H20, ..., H23), all the other Holt devices are in the role of attackers (the Excaliburs are not used in this case). We select one of the two receivers (P1 or P2). We run all the stages of the algorithm with the a set of hyper-parameters to be evaluate. We repeated this procedure for all possible values of voting thresholds (0-127).
  
  We calculate the false-negative-rate and the false-positive-rate (FNR \& FPR respectively) as a function of the threshold. We then find the equal error rate (EER), the rate at which the FNR equals the FPR. The EER is the metric we use for comparing different hyper-parameters. For some cases we supply further plots to aid in qualitative evaluation.
  
  In all cases we used a train-test split of 60\%-40\%.
  
\subsection{Identifying a Rogue Transmitter}
  The results of the intruder detection tests described in the previous section are summed up in the form of a box plot in Figure \ref{fig:rogue_transmitter_results}. \textcolor{gray}{(the following is an explanation of a box plot. unnecessary?)} The horizontal line in the middle of the box and the number written next to it indicate the median. The lower and top boundaries of the box indicate the 1st and 3rd quartiles respectively. The horizontal lines outside of the box (the whiskers) are placed on the minimum and maximum values. On top of the box plot, we plotted the individual points, that make up the box plot.
  The y axis shows the false alarms rate (FAR). This gives a more concrete sense to the number. The FAR calculated by multiplying the EER by the maximum message rate. The false alarm rate is the inverse of mean time between failures.
  
  \[FAR = \frac{1}{MTBF} = EER \cdot \frac{100 \sfrac{kbits}{sec}}{36bits}\]
  
  Each word lasts 36 bit times, because the protocol mandates a minimum inter-word gap of at least 4 bit times.
  
  Note that since the FAR is linear in the FNR, and thus in the EER, we can discuss the bar graph as though it displays EER when giving a qualitative analysis.
  
\subsection{Results and Discussion}
  The results are displayed by feature set. We can observe that intruder detection yields the best results in term of EER when operating in the time domain, without extracting features. Both the median and the spread of the values is low. The EER values for the Scission data set are slightly more spread out, and the median is greater. The poly and the domain specific data sets continue in this trend. It is interesting to note, that even though the domain specific data set performs poorly when guarding some transmitters, for others it achieves an EER of 0.
  
  The data points are color coded by device type. We see that in all feature sets, the Excalibur devices achieve an EER of 0. This might be due to the fact in our experiments, only Holts are used as simulated attackers. When an Excalibur is designated as the legitimate device, the algorithm has to differentiate between two different types of devices. In contrast, when a Holt is the legitimate device, the algorithm has to differentiate between transmitters that are much more similar, either different ports of the same chip (H10 vs. H11), different instances of the same chip (H10 vs. H12, H13) or different instances of the same board model (H10 vs. H2x).
  
  \begin{figure}[t]
    \centering
    \includegraphics[width=1.0\linewidth, angle=0]{Graphs/far_feature_sets_w_domain_w_median.png}
    \caption{Identifying a rogue transmitter}
    \label{fig:rogue_transmitter_results}
  \end{figure}
  
  Before discussing Figure \ref{fig:rogue_transmitter_results} as a whole, we will examine two experiments in order to get a better feel for the meaning of each test. Figure \ref{fig:detection_easy_example} show the FNR and the FPR as a function of the threshold value for the E1 as a guarded device, with P1 as a receiver. The Scission set is used. For comparison, Figure \ref{fig:detection_difficult_example} is a representation H21 as a guarded transmitter and P1 as a receiver. It is apparent that this is a more challenging case. In Figure \ref{fig:detection_easy_example}, the FNR and the FPR do not intersect. There is a wide range of thresholds, for which both an FNR of 0 and an FPR of can be achieved simultaneously. In contrast, in Figure \ref{fig:detection_difficult_example} there is only a narrow range of thresholds for which both error rates are small, and the EER is greater than zero.
  
  In an attempt to understand where the difference is coming from, we plot the TPR and the TPR for each type of segment individually, before the voting stage. The graphs are shown in Figures \ref{fig:segments_easy_example} \& in Figure \ref{fig:segments_difficult_example}. We see that in \ref{fig:segments_difficult_example} the errors are not equally divided among all segment types, as opposed to \ref{fig:segments_easy_example}.

  \begin{figure}[t]
    \centering
    \includegraphics[width=1.0\linewidth, angle=0]{Graphs/scission_e1_p1_fnr_fpr.png}
    \caption{FNR and FPR for E1 as guarded as a function of the threshold}
    \label{fig:detection_easy_example}
  \end{figure}
  
  \begin{figure}[t]
    \centering
    \includegraphics[width=1.0\linewidth, angle=0]{Graphs/scission_h21_p1_fnr_fpr.png}
    \caption{FNR and FPR for H21 as guarded as a function of the threshold}
    \label{fig:detection_difficult_example}
  \end{figure}
  
  \begin{figure}[t]
    \centering
    \includegraphics[width=1.0\linewidth, angle=0]{Graphs/segments_tpr_tnr_e1_p1.png}
    \caption{Per segment FNR and FPR for E1 as guarded}
    \label{fig:segments_easy_example}
  \end{figure}
  
  \begin{figure}[t]
    \centering
    \includegraphics[width=1.0\linewidth, angle=0]{Graphs/segments_tpr_tnr_h21_p1.png}
    \caption{Per segment FNR and FPR for H21 as guarded}
    \label{fig:segments_difficult_example}
  \end{figure}
  
  \color{gray}
    graphs of:

    port-to-port sensitivity

    Excalibur as guarded
 
    Holt as guarded
   \color{black}
 
\subsection{Feature-Selection Revisited}

\subsection{Are we sensitive to the transmitter or receiver}
  In this set of experiments we replace the receiver instead of the transmitter. The results are shown in Figure \ref{fig:receiver_results}.
  
  \begin{figure}[t]
    \centering
    \includegraphics[width=1.0\linewidth, angle=0]{Graphs/far_feature_sets_receiver_w_domain_w_median.png}
    \caption{Identifying an eavesdropper}
    \label{fig:receiver_results}
  \end{figure}
  
\subsection{Adding another receiver}
  In this set of experiments only the Excaliburs are transmitting. One Excalibur transmits data. This is the guarded transmission. We then add one Holt receiver. We check whether the two states can be told apart.
  
  The results are shown in Figure \ref{fig:load_results}. \textcolor{gray}{I'm not sure we should include this results in our paper, except maybe to show it doesn't work? The mead EER for time feature set is 0.38, and for domain specific it's 0.49 == random.}
  
  \begin{figure}[t]
    \centering
    \includegraphics[width=1.0\linewidth, angle=0]{Graphs/far_feature_sets_load_w_domain_w_median.png}
    \caption{Identifying a receiver + line switch}
    \label{fig:load_results}
  \end{figure}
  

\subsection{Sensitivity to message content in data set}

\section{Conclusions}

\bibliographystyle{IEEEtranS}
\bibliography{biblio} 

\end{document}